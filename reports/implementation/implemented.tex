Before evaluating each functional requirement and quality requirement, table \ref{tab:implemented_reqs} lists what requirements that we as developers have implemented. That does not imply that every requirement is working, but rather what has been implemented and needs testing. 

The requirements that haven't been implemented, will not be tested. We refer to the \emph{requirements description document} for more in depth information on the requirements

\begin{table}[H]
	%\resizebox{\textwidth}{!}{
	\begin{center}
	\rowcolors{0}{white}{tableShade}
	\begin{tabular}{p{2cm} | p{2cm} | p{4cm}}
    	\hline
		\textbf{ID} 			& 	\textbf{Priority}	&	\textbf{Implemented}\\ 
		\hline
		FR1			&	High				&	Yes						\\
		FR2			& 	High				& 	Yes						\\
		FR3			& 	High 				& 	Yes						\\
		FR4			&	High				& 	Yes 					\\
		FR5			& 	High				& 	Yes						\\
		FR6			& 	High				& 	Yes						\\
		FR7			& 	High				& 	Yes						\\
		FR8			&	High				&	Yes						\\
		FR9			&	High				& 	Yes						\\
		FR10		&	High				& 	\textbf{No} 			\\
		FR11		&	High				&	Yes\*					\\
		FR12		&	High				&	Yes						\\
		FR13		&	Medium				&	Yes						\\
		FR14		&	Medium				& 	Yes						\\
		FR15		&	Medium				& 	Yes						\\
		FR16		&	Low					&	\textbf{No}				\\
		FR17		&	High				&	Yes						\\
		FR18		&	Medium				& 	\textbf{No}				\\
		FR19		&	Medium				&	Yes						\\
		FR20		&	Medium				&	Yes						\\
		FR21		&	High				&	Yes						\\
		\hline
    \end{tabular}
\end{center}
	\caption{Implemented functional requirements}
	\label{tab:implemented_reqs}
\end{table}

