\subsection{Logical View}

\begin{table}[H]
	\resizebox{\textwidth}{!}{
	\rowcolors{0}{white}{tableShade}
	\begin{tabular}{p{5cm} | p{12cm}}
    	\hline
		\textbf{Objective}		&	Describes the functionality of the system with a modeling language \\															
		\textbf{Notation}		& 	Implied by the chosen modeling language \\ 
		\textbf{Stakeholder}	& 	Testers, evaluators and developers	\\
		\textbf{Selection}		& 	Class diagrams \\
		\hline
    \end{tabular}
	\label{tab:log_view}}
	\caption{Logical View}
\end{table}

\subsection{Process View}

\begin{table}[H]
	\resizebox{\textwidth}{!}{
	\rowcolors{0}{white}{tableShade}
	\begin{tabular}{p{5cm} | p{12cm}}
    	\hline
		\textbf{Objective}		&	Describes non-functional requirements constraining performance, 
									concurrency, integrity and availability. \\															
		\textbf{Notation}		& 	State, Activity and sequence diagrams \\ 
		\textbf{Stakeholder}	& 	Evaluators and developers	\\
		\textbf{Selection}		& 	Sequence diagrams \\
		\hline
    \end{tabular}
	\label{tab:process_view}}
	\caption{Process View}
\end{table}

\subsection{Development View}

\begin{table}[H]
	\resizebox{\textwidth}{!}{
	\rowcolors{0}{white}{tableShade}
	\begin{tabular}{p{5cm} | p{12cm}}
    	\hline
		\textbf{Objective}		&	Describes the system by modules and subsystem diagrams, 
									showing the ‘export’ and ‘import’ relationships. \\															
		\textbf{Notation}		& 	Booch \\ 
		\textbf{Stakeholder}	& 	Evaluators and developers	\\
		\textbf{Selection}		& 	None at this iteration \\
		\hline
    \end{tabular}
	\label{tab:dev_view}}
	\caption{Development View}
\end{table}

\subsection{Physical View}

\begin{table}[H]
	\resizebox{\textwidth}{!}{
	\rowcolors{0}{white}{tableShade}
	\begin{tabular}{p{5cm} | p{12cm}}
    	\hline
		\textbf{Objective}		&	Describes the the non-functional requirements of 
									the system such as availability, reliability 
									(fault-tolerance), performance (throughput), and 
									scalability.  \\															
		\textbf{Notation}		& 	Physical blueprints \\ 
		\textbf{Stakeholder}	& 	Not relevant, available if needed\\
		\textbf{Selection}		& 	None, not relevant \\
		\hline
    \end{tabular}
	\label{tab:Physical view}}
	\caption{Physical View}
\end{table}

\subsection{Scenarios}

\begin{table}[H]
	\resizebox{\textwidth}{!}{
	\rowcolors{0}{white}{tableShade}
	\begin{tabular}{p{5cm} | p{12cm}}
    	\hline
		\textbf{Objective}		&	To use the other views and create scenarios which 
									works towards validating the architectural choices	\\															
		\textbf{Notation}		& 	Scenarios, Scenario diagrams \\ 
		\textbf{Stakeholder}	& 	All\\
		\textbf{Selection}		& 	Scenarios \\
		\hline
    \end{tabular}
	\label{tab:scenario_view}}
	\caption{Scenarios}
\end{table}