\subsection{Model View Controller}
The Model View Controller pattern (MVC) is a way of dividing an application
into clear and separated responsibilities. MVC is not perfectly suited to 
games, but we will be using some of the ideas behind it to achieve modularity.

The problem with using MVC in a game is that views, i.e. a camera, are closely
related to the game itself, which represents the model.  Controller is also 
not easily fitted in this particular game, due to the way XNA handles IO.

Our solution is to separate out as much of the view as possible in a separate
renderer module.  This will need some knowledge of the game, but we hope to 
minimize this by creating general methods which may be applied in several 
contexts.

\subsection{Event-driven programming}
While event-driven programming is not strictly a pattern, it is such a big 
part of our architecture we feel it should be mentioned here.

The thinking behind our version of this is to split functionality in the game
into small components which know how to handle a small set of events, and 
achieving more complex features through interaction between these components.

\subsection{Component-based entities}
Entities are anything and everything which appears in-game.  This includes the
player, creeps and the level.  We hope to make these parts as interchangeable 
as possible by using a component-based system for entities.  The idea is that
functionality can be composed of small reusable components which can interact.
This is the approach taken in Unity, one of the most popular game-engines for 
the Windows and Mac platforms\cite{unity}.