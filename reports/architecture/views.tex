\subsection{Logical View}

\begin{figure}[H]
	\begin{center}
		\includegraphics[scale=0.75]{graphics/DataManagementView}
	\end{center}
        \caption{Box and arrow diagram of game classes}
\end{figure}

The box and arrow diagram shows which modules and components that will be
used to implement the design. XNA assumes a Game class as the entry point of
the application, so this will be where everything starts. Aside from that,
we clearly need a DataManager to facilitate a data-driven application. The
idea here is that the DataManager will be able to import all kinds of data that
may be required by other components of the game, such as sprites, hero statistics,
levels and so on. It should also be able to store and retrieve a highscore list.

The GameState stack is how the game keeps track of what it should be doing right
now. For example, assuming the player pushes Start to pause the game, what should
happen is that a Paused state is pushed onto the stack, and then popped whenever
the player is done pausing. This is where menus and the like are handled.

We have decided that a Renderer is desirable, so that we can switch around on how
the graphics are handled at a later date. To facilitate this, we implement a
uniform interface for how an object can get itself rendered to the screen, and
an instance of a Renderer of some sort is passed down the call-chain down to the
entities that wish themselves to be rendered. In other words, this enables us to
not have to define how objects are Rendered in the objects themselves, and should
enable us to easily change graphical styles and the likes late in the development
cycle.

Movable, collidable objects are Entities. Heros and creeps are subclasses
of this class, as is Particle, which has a slightly different behaviour.

\subsection{Player Shoots}
\begin{figure}[H]
	\begin{center}
		\includegraphics[scale=0.75]{graphics/PlayerShoots}
	\end{center}
        \caption{Player shoots}
\end{figure}

The essential of the situation is that a player checks its input, and finds the shoot button to be
pressed. This causes it to create a PlayerShootsEvent, which is pushed to the
EventQueue. Next iteration of the game update loop causes this event to be
popped and processes, spawning a Projectile in the end. When this projectile
collides with something, it can create a CollisionEvent causing the game
to update health and draw a nice explosion or something similar.

\subsection{Scenarios}

\begin{figure}[H]
	\begin{center}
		\includegraphics[scale=0.75]{graphics/scenario_1_StartingGame}
	\end{center}
        \caption{Game is started}
\end{figure}

Scenario 1 details which components that interact when a game is started. Scenario
2 details what takes place once a player dies, and achieves a new high score. 

Do note that "intro state" is not implemented as of 08.04.11. It should be easy to 
implement. 

\begin{figure}[H]
	\begin{center}
		\includegraphics[scale=0.75]{graphics/scenario_2_PlayerDiesHighScore}
	\end{center}
        \caption{Player dies, and achieves a highscore}
\end{figure}
