\subsection{Availability Tactics}
\begin{itemize}
	\item Handle battery failure of Xbox-controller by pausing game and
	allowing resuming later.
\end{itemize}

\subsection{Modifiability Tactics}
\begin{itemize}
	\item The architecture should be as modular as possible, with 	
	the responsibilities of each module clearly delineated.  This should
	localize changes to the system to the greatest possible degree.
	\item By using a data-driven approach, the architecture should be well
	suited for fast changes.
	\item Component-based game logic.  Each entity in the game, be it the
	player or a puff of smoke, should be made up from small components with
	clear responsibilities.  This should allow quick changes to gameplay at
	a late date, and allow us to add new functionality without affecting
	existing code.
\end{itemize}

\subsection{Usability Tactics}
\begin{itemize}
	\item The game should have a clean and simple graphical style, and should
	not confuse the player with needless complexity.
	\item The interface should follow Microsofts UI-guidelines as much as 
	possible.
	\item By using the Model-View-Controller pattern we hope to achieve a 
	simple architecture to drive the application.
\end{itemize}

\subsection{Performance Tactics}
\begin{itemize}
	\item Profile then optimize.
	\item Decouple rendering and game logic.
\end{itemize}

\subsection{Security Tactics}
\begin{itemize}
	\item Not relevant to this project
	\item Non-networked games can be considered "trusted"
\end{itemize}

\subsection{Testability tactics}
\begin{itemize}
	\item The game should monitor internal state, and allow reporting this
	in a debug-mode for developers.
	\item Unit-tests should be employed as extensively as possible
\end{itemize}