The information form the course staff was in our view a bit lacking, and as such
we missed the first meeting with our assigned group.  This caused some issues
as we could not find an open slot in our timetables until 4 days before the
delivery deadline.  Strapped for time, our understanding of the architecture
and ATAM process suffered accordingly.

We also feel that ATAM is not very flexible, and requires a specific
way of documenting requirements, scenarios and architecture to work at
its best. It was still a useful exercise in that it raised useful discussions
about architecture for both projects.

The best thing we can say about ATAM so far is that it is a learning process,
and both groups have learned more about how to write architecture and
requirement documents in such a way as to more easily find strengths
and weaknesses to architectures.

