\begin{table}[H]
	%\resizebox{\textwidth}{!}{
	\rowcolors{0}{white}{tableShade}
	\begin{center}
		\begin{tabular}{| c | p{10cm} | }
    		\hline
			S1 & The choice to disregard Android usability guidelines affects
			     usability negatively.\\
			S2 & Using a data-driven approach to the UI makes modifications
			     easier.\\
			S3 & Consciously considering code quality positively affects
			     development time and modifiability. (This point is not
			     covered by any of the scenarios we analyzed carefully,
			     but we feel it is important enough to warrant a mention
			     here in any case.)\\
			S4 & Using Factory Pattern should simplify adding new features
			     such as additional powerups and traps\\															
			S5 & MVC is well supported on the Android platform, and the
			     decision to use it should lighten developer load and
			     make the code cleaner and more modifiable.\\
			S6 & Using an already existing framework (Sheep) should
			     positively affect code modifiability by offering simple
			     points to hook features into.  It may however become a
			     limitation if it turns out to be a bad match for the
			     planned game.\\
		    S7 & Limiting the options in menu screens should keep understanding
		         the game easy, which positively affects usability.\\
		    S8 & The observer-observed pattern promotes decoupling of
		         cause and effect. By using it the architecture should be
		         more easily modifiable.\\
			\hline
    	\end{tabular}
	\end{center}
	\label{tab:sensetivityPoints}
	\caption{Sensitivity Points}
\end{table}
